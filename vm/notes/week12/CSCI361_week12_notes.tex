\documentclass[12pt]{article}

\usepackage[utf8]{inputenc}
\usepackage{latexsym,amsfonts,amssymb,amsthm,amsmath,graphicx}
\usepackage[shortlabels]{enumitem}

\setlength{\parindent}{0in}
\setlength{\oddsidemargin}{0in}
\setlength{\textwidth}{6.5in}
\setlength{\textheight}{8.8in}
\setlength{\topmargin}{0in}
\setlength{\headheight}{18pt}



\title{CSCI 361 Week 12 Notes}

\author{
  Kyle Krstulich
}

\begin{document}

\maketitle

\section*{Virtual Machine part 2}

We will be implementing the quadratic formula
\begin{verbatim}
x_1 = (-b + sqrt(pow(,b2) - 4ac)) / 2a
x_2 = (-b - sqrt(pow(,b2) - 4ac)) / 2a
\end{verbatim}

Functions can be a problem, you need to jump to subroutine code. This is an integer only computer,
so rational numbers will be a problem. We need a mechanism to context switch between functions. The
function doesn't know anything about the caller.

We need to save the state of the caller.
\begin{itemize}
  \item \textbf{Local}: Stores local variable ID callers environment(function). Take the local
    memory segment and save it somewhere. When you jump to functions you jump to a different local
    memory segment. From caller local to callee local. 

  \item \textbf{Argument}: Every function has arguments. The argument segment will store these
    arguments. Caller puts values in argument segment, a deep level of recursion can overflow the
    stack by filling it with argument segments.

    To pass arguments to subroutine: push stack. To return pop stack.

  \item \textbf{THIS}: Maintains state of stack.

  \item \textbf{THAT}: Maintains state of stack.
\end{itemize}

\subsection*{Homework}
New VM keywords to be implemented in chapter 8.
\begin{itemize}
  \item Flow control: goto label, if-goto label

  \item Function Use: call f m, function f m, return

    \begin{itemize}
      \item f is just a label, and address in RAM.
      \item m is the number of arguments that you pass into the function.
      \item n is start of function in VM, n is the number of local variables.
      \item return just makes top of stack the result.
    \end{itemize}
\end{itemize}


\end{document}


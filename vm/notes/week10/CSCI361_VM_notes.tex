\documentclass[12pt]{article}

\usepackage[utf8]{inputenc}
\usepackage{latexsym,amsfonts,amssymb,amsthm,amsmath,graphicx}
\usepackage[shortlabels]{enumitem}

\setlength{\parindent}{0in}
\setlength{\oddsidemargin}{0in}
\setlength{\textwidth}{6.5in}
\setlength{\textheight}{8.8in}
\setlength{\topmargin}{0in}
\setlength{\headheight}{18pt}



\title{CSCI 361 Virtual Machine Notes}

\author{
  Kyle Krstuich
}

\begin{document}

\maketitle


\section*{History}

The model of computation in the 1990's
\begin{verbatim}
  High level language
        | Compiler
  Machine Language
        | Assembler
  Op codes/Binary
        | control logic
  Hardware
\end{verbatim}

Later on the model was changed.
\begin{verbatim}
  High level language
        | Virtual machine translator
  Runtime Virtual Machine
        | Stack based computing
  Machine Language # can also have a language compile directly to machine code
        | Assembler
  Op codes/Binary
        | control logic
  Hardware
\end{verbatim}
Run time virtual machine is a lower lever application that has its own machine
code that you compile your higher level language to. This makes it so that you
don't Hae to write you code specific to an architecture. This however, is slow. 

This week, we assume that we have code for the virtual machine. Me make a
Python program translator for machine language using stack based computing. 

\clearpage

\section*{Building a VM}
VM supports elementary stack operations. LIFO data structure.
\begin{itemize}
  \item
    Pop() off

  \item
    Push() onto
\end{itemize}
A stack is usually a segment of RAM with a stack pointer(SP) just above the
stack. This is useful because you can push in constant time.

\begin{verbatim}
  push(x):
    stack[SP] = x
    SP++

  pop():
    SP--
    return stack[SP]
\end{verbatim}

Stacks are lossy or lose data at the time of removal. While RAM is lossy at
time of input, because RAM overwrites values, while the stack 'forgets' values.

In other classes I played with the top of the stack. However the stacks in this
class work from the bottom. This is because when the stack grows it does down
in memory where there is more room.

\clearpage

\section*{Stack Arithmetic}
Pop operands to preform operations. Push result back.

Operands needed to implement:
\begin{itemize}
  \item
    ADD; pops top two values off stack, add them, and push the result. Takes
    the deeper value first.
    \begin{verbatim}
  a = pop()
  b = pop()
  push(b+a)
    \end{verbatim}
  \item
    SUB; pops top two values off stack, subtracts them, and then push the
    result. Takes the deeper value first.
    \begin{verbatim}
  a = pop()
  b = pop()
  push(b-a)
    \end{verbatim}
  \item
    NEG *
  \item
    EQ
  \item
    GT
  \item
    LT
  \item
    AND
  \item
    OR
  \item
    NOT 8
\end{itemize}
* are urinary operators.

\subsection*{Class example}
\begin{verbatim}
  operation D = (2-x) + (y+5)
  x=5, y=3

  1. push(2) ->            |2|
                           |SP|

  2. push(@x) ->           |2|
                           |x|
                           |SP|

  3. SUB  # 2-x ->         |-3|
                           |SP|

  4. push(@y) ->           |-3|
                           |y|
                           |SP|

  5. push(5) ->            |-3|
                           |y|
                           |5|
                           |SP|

  6. ADD # y+5 ->          |-3|
                           |8|
                           |SP|

  7. ADD # (2-x) + (y+5) -> |5|
                            |SP|
\end{verbatim}
Binary operations require 2 arguments such ass ADD. Unary operators only
require 1 argument such as NOT. The structure for the operations is very
different. Pops are not a direct memory access, it is an indirect address.

\section*{More History}
In the 90's and early 2000's compilers did not have a virtual machines, and were tightly coupled to
the hardware architecture. There were many different compilers, free and paid.\\

Modern compilers compile the high level code to bytecode, which is read by the virtual machine, to
then translate that bytecode into assembly, then assembled into binary. This is also called a
two-stage compilation process.\\

\textit{For Tuesday build push/pop to the D register.}\\

\subsection*{Another Class Example}
\begin{verbatim}
  (x<7) or (y=8)
  x=12, y=8

  1. push(x) ->             |x|
                            |SP|

  2. push(7) ->             |x|
                            |7|
                            |SP|

  3. LT # x < 7 ->          |FALSE|
                            |SP|

  4. push(y) ->             |FALSE|
                            |y|
                            |SP|

  5. push(8) ->             |FALSE|
                            |y|
                            |8|
                            |SP|

  6. EQ # does y = 8? ->    |FALSE|
                            |TRUE|
                            |SP|

  7. OR # (x>7) or (y=8) -> |TRUE|
                            |SP|
\end{verbatim}

\subsection*{The VM File}
.vm files is what you read in. The output will be hack assembly. VM files are on the website. There
are three formats for commands in the VM file. 
\begin{enumerate}[1.]
  \item
    Command
  \item
    Command argument

  \item
    Command argument arguments
  
\end{enumerate}

\subsection*{Memory Access Commands}
We need to say
\begin{itemize}
  \item
    Push segment index

  \item
    Pop segment index
\end{itemize}

Where a segment is:
\begin{itemize}
  \item
    Argument *: used to store activation record
  \item
    local *: also used to store activation record
  \item
    static -
  \item
    constant +: for direct push of literal, is a 'virtual' segment.

  \item
    this/that * 

  \item
    pointer *

  \item
    temp +

\end{itemize}
* for each function

- for all functions in VM file

+ for \textit{all} functions
\end{document}

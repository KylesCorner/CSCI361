\documentclass[12pt]{article}

\usepackage[utf8]{inputenc}
\usepackage{latexsym,amsfonts,amssymb,amsthm,amsmath,graphicx, tikz,xparse}
\usetikzlibrary{positioning}
\usepackage{algorithm,algorithmicx,algpseudocode}
\usepackage[shortlabels]{enumitem}

\setlength{\parindent}{0in}
\setlength{\oddsidemargin}{0in}
\setlength{\textwidth}{6.5in}
\setlength{\textheight}{8.8in}
\setlength{\topmargin}{0in}
\setlength{\headheight}{18pt}

\newenvironment{SafeTikz}
  {\noindent\ignorespaces%
   \begin{center}%
   \begin{tikzpicture}[node distance=0pt]}
  {\end{tikzpicture}%
   \end{center}%
   \ignorespacesafterend}

% Macro to draw a vertical stack from comma-separated values
\newcommand{\drawstack}[1]{%
  \begin{SafeTikz}
    \foreach \elem [count=\i] in {#1} {
      \node[draw, minimum width=2cm, minimum height=1cm, anchor=south] (n\i) at (0, \i - 1) {\elem};
    }
  \end{SafeTikz}
}

\title{CSCI 361 Week 13 Notes}

\author{
  Kyle Krstulich
}

\begin{document}

\maketitle

\section*{Announcements}
Chapter 8 will be the last major programming assignment. May 1st will be exam 2, while the final is
on the syllabus, Tuesday on finals week.

Helper functions due Thursday, while next week we will cover the harder sections.

\section*{Function Calls}

High level language:
\begin{verbatim}
  if (cond)
    s1
  else
    s2
\end{verbatim}

In order to translate this to a VM language. (not pythonic indentation)
\begin{verbatim}
  ~cond (n = not)
  if-goto L1
    s1
    goto L2
  label L1
    s2
  label L2
\end{verbatim}

To call a function you must first push all arguments to stack.

\subsection*{Call Stack and Recursion}
\begin{verbatim}
  subroutine a:
    call b
    call c
  subroutine b:
    call c
    call d
  subroutine c:
    call d
  subroutine d:
    null
\end{verbatim}

Trace:
\begin{verbatim}
  call a
    call b
    push a
      call c
      push b
        call d
        push c
        return d
      return c
      pop c
    call d
      push b
      return d
    pop b
  call c
    push a
    call d
      push c
      pop c
    pop a
  pop a

\end{verbatim}

\subsection*{Function POV}

Calling function POV
\begin{itemize}
  \item push arguments to stack
  \item Static segment is set based on filename. 
  \item 'call' function (jump to that code)
  \item return value. Top of stack contains the function return value.
  \item memory segments; argument, local, static, this, that, pointer. Are as they were before the
    call. Function call does not mutate the memory segments.
\end{itemize}

POV of called function
\begin{itemize}
  \item Local segment initialized/reserved set to 0 (in hack). 
  \item Static segment is set based on filename. 
  \item Working stack appears to be empty. this, that, pointer, temp are undefined. This defines a
    pristine workspace for the function.
  \item will push return values to top of stack.
\end{itemize}

\section*{Function Call Implementation}
At the point of calling a function:
\drawstack{
  {LCL var n-1},
  {$\cdots$},
  {LCL var 1},
  {LCL var 0},
  {saved That},
  {saved THIS},
  {saved ARG},
  {saved LCL},
  {return address},
  {arg n-1},
  {$\cdots$},
  {arg 1},
  {arg 0},
  {Working Stack}
}
\begin{itemize}
  \item \textbf{Return Address}: uses push label. Lets us know where to come back to after function
    call is finished. Is a ROM address.
  \item \textbf{Frame}: the LCL, ARG, THIS, THAT, and return address.
  \item \textbf{Local Variables}: LCL var n
  \item \textbf{LCL, ARG, THIS, THAT} uses pushMem()
\end{itemize}

Overall handing a function call:
\begin{itemize}
  \item Save the return address
  \item Save the callers segment pointers (LCL, ARG, THIS, THAT)
  \item Reposition ARG (for the callee) argmuments are located after the caller working stack.
  \item Reposition LCL (for the callee)
  \item Go to execute the callee's code (@function, 0;JMP)
\end{itemize}


\end{document}


